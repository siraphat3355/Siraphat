% add to Table of content
\addcontentsline{toc}{part}{ABSTRACT}

% set 0 indentation
\setlength{\parindent}{0in}
% set paragraph space = 1 space
\setlength{\parskip}{1em}
% set line space = 1.5
\setlength{\baselineskip}{1.6em}

\begin{center}
  \fontsize{14}{17}\selectfont{\textbf{
      ABSTRACT
    }}
\end{center}
\vspace{2em}

Falls are a global public health problem. Falls happen to people of all ages, especially on the elderly. Throughout the last decade, we have seen improvements in fall detection system due to technology development and the revolution of deep learning. However, using vibration signal analysis can compensate the weakness and also overcomes the drawbacks associated with the traditional system, and this is a novel idea that needs to be studied further. This thesis studies the embedded system and design space for unsupervised anomaly detection model using modern deep learning best practices. The performance and effectiveness of this system to immediately send alert message to user via LINE apllication when abnormal events occur. Accordingly, this study can help the home residents when an anomolous event or falling down event is occurring.

\textbf{Keywords:} falling down, anomaly detection, deep learning, Transformer model.